Das neutrale Pion $\pi^{0}$ besteht aus einem Quark-Antiquark-Paar und gehört damit zu den Mesonen.
Genauer lässt sich das $\pi^{0}$ als eine Überlagerung zweier quantenmechanischer Zustände, bestehend aus $u$ und $d$ Quarks und den entsprechenden Antiquarks, beschreiben:
\begin{align}
\ket{\pi^{0}} = \frac{1}{\sqrt{2}}\left(\ket{u\bar{u}}-\ket{d\bar{d}}\right) \label{eq:pi0state}
\end{align}
Mit einer Masse von $m_{\pi^{0}} = \left(134,9770\pm0,0005\right) \rm{MeV}/c^{2}$ \cite{book:pdg} stellt das $\pi^{0}$ das leichteste Meson dar.
Ein ${\pi^{0}}$ zerf{ä}llt zu $\left( 98,823\pm0,034\right)\%$ nach einer mittleren Weglänge von ${\it c\tau} = (25,5\pm0,5)$nm \cite{book:pdg} in zwei Photonen.
\newline
%%%%%%%%%%%%%%%%%%%%%%%%%%%%%%%
Beim ALICE Experiment in Kern-Kern-Kollisionen werden unter anderem direkte Photonen untersucht.
Als direkte Photonen werden solche Photonen bezeichnet, die in der Kollision entstehen und nicht aus Zerfällen stammen.
Direkte Photonen können allerdings nicht direkt bestimmt werden.
Stattdessen werden alle Photonen, die produziert wurden, gemessen und die Anzahl Photonen aus Zerfällen werden von der Anzahl aller Photonen subtrahiert.
Dazu muss zuerst die Anzahl der Photonen, die aus Zerfällen kommen, bestimmt werden, wozu wiederum die \textit{Yields} von Teilchen extrahiert werden müssen, die in Photonen zerfallen.
Aufgrund der hohen Produktionsrate von $\pi^{0}$ in Kern-Kern-Kollisionen und der hohen Zerfallswahrscheinlichkeit in zwei Photonen stellen Photonen aus $\pi^{0}$-Zerfällen den größten Anteil von Zerfallsphotonen.
\newline
Direkte Photonen können auch in pp-Kollisionen betrachtet werden.
In pp-Kollisionen gibt es dabei ebenfalls eine hohe Produktionsrate von $\pi^{0}$, weshalb die Analyse von $\pi^{0}$, für die Bestimmung von direkten Photonen essentiell ist.
Die Analyse von $\pi^{0}$ in pp-Kollisionen liefert somit eine direkte Referenzgröße in Form des \textit{Yields} von $\pi^{0}$, als auch eine Referenz für direkte Photonen.
Das Verhältnis der Produktionsraten von $\pi^{0}$ in Kern-Kern-Kollisionen gegenüber der Produktionsraten von $\pi^{0}$ in pp-Kollisionen kann so beispielsweise Aufschluss geben auf den Energieverlust von Teilchen innerhalb des QGP.
Deshalb werden in dieser Arbeit die Produktion von $\pi^{0}$ in pp-Kollisionen analysiert.
\newline
Gemessen werden bei ALICE allerdings nicht $\pi^{0}$ direkt, sondern nur die Photonen, aufgrund der kurzen Lebensdauer des $\pi^{0}$.
Deshalb müssen $\pi^{0}$ über Messungen der Photonen rekonstruiert werden.
Durch geeignete Messungen können Energie und Position der beiden Photonen bestimmt werden.
Durch die Information über die Position der Photonen kann auch der Zerfallswinkel zwischen den beiden Photonen $\theta_{\gamma\gamma}$ bestimmt werden.
Die Energien $E_{\gamma1}$ und $E_{\gamma2}$ der beiden Photonen sowie der Zerfallswinkel $\theta_{\gamma\gamma}$ werden benötigt, um die invariante Masse $m_{\text{inv}}$ eines $\pi^{0}$ zu berechnen.
Für diese gilt:
\begin{align}
m_{\text{inv}} &= \sqrt{2E_{\gamma\it{1}}E_{\gamma\it{2}}(1-\cos\left( \theta_{\gamma\gamma}\right) )} \label{eq_invmass}
\end{align}
%Die Bedeutung der invarianten Masse wird in Abschnitt \ref{s3s2} verdeutlicht.
\newline
Neben der Bestimmung der invarianten Masse kann der Impuls der Photonen aufgeteilt werden, in den Transversalimpuls und den Longitudinalimpuls.
Dabei wird in dieser Arbeit nur der Transversalimpuls $p_\text{T}$ des $\pi^{0}$ betrachtet für den gilt:
\begin{align}
p_{T\pi^{0}} &= \sqrt{\left(p_{x1}+p_{x2}\right)^{2} +\left(p_{y1}+p_{y2}\right)^{2}} \label{eq_pt}
\end{align}
Die Indizes x und y beziehen sich auf die Raumrichtungen.
\newline
In einer Kollision entstehen allerdings mehrere $\pi^{0}$ auf einmal.
Die Information, welche Photonen dabei aus welchem Zerfall stammen, geht bei der Messung verloren.
Deshalb werden alle gemessenen Photonen miteinander kombiniert, wodurch einerseits $\pi^{0}$ rekonstruiert werden, andererseits werden aber auch Photonenpaare miteinander kombiniert, die nicht aus einer Zerfallskette stammen, oder nicht aus einem $\pi^{0}$-Zerfall.
Um die Anzahl an Photonenpaaren, die nicht aus einem $\pi^{0}$-Zerfall kommen, abzuschätzen, werden in dieser Arbeit sogenannte Templates aus einer Monte Carlo Simulation verwendet.
Für eine detailliertere Beschreibung von Monte Carlo Simulationen sei an dieser Stelle auf \cite{thesis:Krissy} verwiesen.
Templates beschreiben in dieser Analyse Verteilungen der invarianten Masse, die aus Monte Carlo Simulationen stammen.
Durch das Verwenden von Templates in dieser Arbeit kann das theoretische Verständnis von pp-Kollisionen überprüft und vertieft werden.
Eine genauere Erläuterung der Templates die für diese Arbeit verwendet werden, folgt in den Abschnitten \ref{s3s5s1} und \ref{s3s5s2}.
\newline
Nachdem die theoretischen Grundlagen für die Analyse von $\pi^{0}$ dargelegt wurden, wird in Abschnitt \ref{s2} der experimentelle Aufbau näher erläutert.