Das neutrale Pion $\pi^{0}$ besteht aus einem Quark-Antiquark-Paar und geh\"ort damit zu den Mesonen.
Genauer l\"asst sich das $\pi^{0}$ als eine \"Uberlagerung zweier quantenmechanischer Zust\"ande, bestehend aus $u$ und $d$ Quarks und den entsprechenden Antiquarks, beschreiben:
\begin{align}
\ket{\pi^{0}} = \frac{1}{\sqrt{2}}\left(\ket{u\bar{u}}-\ket{d\bar{d}}\right) \label{eq:pi0state}
\end{align}
Mit einer Masse von $m_{\pi^{0}} = \left(134,9770\pm0,0005\right) \rm{MeV}/c^{2}$ \cite{book:pdg} stellt das $\pi^{0}$ das leichteste Meson dar.
Ein ${\pi^{0}}$ zerf{\"a}llt zu $\left( 98,823\pm0,034\right)\%$ nach einer mittleren Wegl\"ange von ${\it c\tau} = (25,5\pm0,5)$nm \cite{book:pdg} in zwei Photonen.
Durch geeignete Messungen k\"onnen Energie und Position der beiden Photonen bestimmt werden.
Durch die Information \"uber die Position der Photonen kann auch der Zerfallswinkel zwischen den beiden Photonen $\theta_{\gamma\gamma}$ bestimmt werden.
Die Energien $E_{\gamma1}$ und $E_{\gamma2}$ der beiden Photonen sowie der Zerfallswinkel $\theta_{\gamma\gamma}$ werden ben\"otigt, um die invariante Masse $m_{\text{inv}}$ eines $\pi^{0}$ zu berechnen.
F{\"u}r diese gilt:
\begin{align}
m_{\text{inv}} &= \sqrt{2E_{\gamma\it{1}}E_{\gamma\it{2}}(1-\cos\left( \theta_{\gamma\gamma}\right) )} \label{eq_invmass}
\end{align}
%Die Bedeutung der invarianten Masse wird in Abschnitt \ref{s3s2} verdeutlicht.
\newline
Neben der Bestimmung der invarianten Masse kann der Impuls der Photonen aufgeteilt werden, in den Transversalimpuls und den Longitudinalimpuls.
Dabei wird in dieser Arbeit nur der Transversalimpuls $p_\text{T}$ des $\pi^{0}$ betrachtet f\"ur den gilt:
\begin{align}
p_{T\pi^{0}} &= \sqrt{\left(p_{x1}+p_{x2}\right)^{2} +\left(p_{y1}+p_{y2}\right)^{2}} \label{eq_pt}
\end{align}
Die Indizes x und y beziehen sich auf die Raumrichtungen.
\newline
Bei einem Kollisionsexperiment besitzen die beiden einfliegenden Teilchen nur einen Impuls in Strahlrichtung.
Durch Wechselwirkungen in der Kollision besitzen Teilchen, die in der Kollision entstanden sind, hingegen einen Transversalen Impulsanteil.
Daher wird oft von Teilchen, die aus der Kollision kommen, nur der transversale Impulsanteil betrachtet.
\newline
Die Messung von $\pi^{0}$ wird aus mehreren Gr\"unden zur Untersuchung von hochenergetischen Teilchenkollisionen verwendet.
\newline
Zum einen, um die Anzahl direkter Photonen bestimmen zu k\"onnen, da direkte Photonen benutzt werden k\"onnen um die Temperatur des Mediums zu bestimmen.
Als direkte Photonen werden solche Photonen bezeichnet, die in der Kollision entstehen und nicht aus Zerf\"allen stammen.
Um die Anzahl direkter Photonen bestimmen zu k\"onnen, wird die Anzahl an Zerfallsphotonen von der Gesamtzahl aller gemessenen Photonen abgezogen.
Aufgrund der hohen Zerfallswahrscheinlichkeit eines $\pi^{0}$ in zwei Photonen, sowie einer hohen Produktionsrate von $\pi^{0}$ in Teilchenkollisionen, kommt ein Gro{\ss}teil der indirekten Photonen von Zerf\"allen von $\pi^{0}$.
Deswegen wird f\"ur die Bestimmung der Anzahl direkter Photonen eine pr\"azise Messung der $\pi^{o}$ ben\"otigt.
\newline
Zum anderen wird die Anzahl produzierter $\pi^{0}$ von Kern-Kern-Kollisionen verglichen mit Kollisionen, bei denen davon ausgegangen wird, dass dort kein QGP entsteht.
Unter anderem Proton-Proton-Kollisionen werden als ein solches Vergleichssystem benutzt.
Das Verh\"altnis der Produktionsraten abh\"angig vom Transversalimpuls kann so beispielsweise Aufschluss geben auf den Energieverlust von Teilchen innerhalb des QGP.
\newline
Nachdem die theoretischen Grundlagen f\"ur die Analyse von $\pi^{0}$ dargelegt wurden, wird in Abschnitt \ref{s2} der experimentelle Aufbau n\"aher erl\"autert.