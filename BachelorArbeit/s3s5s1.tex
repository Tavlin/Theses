Das Template des Signals kommt aus der Monte Carlo Simulation.
Dabei wird ausgenutzt, dass in der Simulation bekannt ist, welchen Ursprung welches Teilchen hat und welches Teilchen auf das EMCal trifft.
Dadurch wird ermöglicht genau bestimmen zu können, ob ein Photonkandidat aus einem Zerfall stammt oder anderen Prozesse entstammt und ob es sich dabei um ein Photon oder ein konvertiertes Elektron oder Positron handelt.
\begin{figure}[tp]
\centering
\includegraphics[width=.75\linewidth]{PeakTemplateMotivation10_Data_2016.pdf}
\caption{Template des Signals (grün) mit seinen drei Teilkomponenten.
Diese bestehen aus Kombinationen mit zwei Photonen (blau), einem Photon und einem Konversionselektron oder -Positron (gelb) und zwei unterschiedlichen Konversionselektron oder -Positron (grau).}
\label{fig:SigTemp}
\end{figure}
\newline
%Joshuas Problem mit dem Satz?
Abbildung \ref{fig:SigTemp} zeigt das Template des Signals in grün, sowie die Aufteilung des Signals in seine einzelnen Komponenten.
Die Komponenten setzen sich aus den drei möglichen Kombinationen von Photonenkandidaten zusammen.
Zum einen aus Photonenkandidaten aus Photonen, in der Abbildung als $\gamma$ bezeichnet und zum anderen aus Photonenkandidaten aus einem Elektron oder Positron, die durch die Konversion  eines Photonen entstanden sind.
Letztere werden in der Abbildung durch $\gamma_\text{conv}$ symbolisiert.
\newline
In blau sind die Kombinationen aus zwei Photonen ($\gamma\gamma$) dargestellt, in gelb die Kombination aus Photon und Elektron oder Positron ($\gamma\gamma_\text{conv}$) und in grau die Kombination aus Konversionselektron oder -Positron miteinander ($\gamma_\text{conv}\gamma_\text{conv}$).
\newline
Die Abbildung zeigt außerdem, wie zuvor angesprochen, dass auch bei einer invarianten Masse um $0\,05 \text{ GeV}/c^{2}$ Signal vorliegt.
Der Anteil des Signals um diese invariante Masse besteht dominant aus zwei konvertierten Photonen.
Genau dieser Teil des Signals wird nicht durch die Standardmethode berücksichtigt.
Durch das Berücksichtigen in der Analyse mit Hilfe der Templates wird einer geringere statistische Unsicherheit erwartet.