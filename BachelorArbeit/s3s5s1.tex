Das Template des Signals wird mit Hilfe der Information der Monte Carlo Simulation erstellt.
Dabei wird ausgenutzt, dass in der Simulation bekannt ist, welchen Ursprung welches Teilchen hat und welches Teilchen auf das EMCal trifft.
Dadurch wird ermöglicht, genau bestimmen zu können, ob ein Photonkandidat aus dem Zerfall eines $\pi^{0}$ oder einem anderen Prozess stammt und ob es sich dabei um ein Photon, ein konvertiertes Elektron oder Positron, oder ein anderes Teilchen handelt.
\begin{figure}[tp]
\centering
\includegraphics[width=.75\linewidth]{PeakTemplateMotivation10_Data_2016.pdf}
\caption{Template des Signals (grün) mit seinen drei Teilkomponenten.
Diese bestehen aus Kombinationen mit zwei Photonen (blau), einem Photon und einem Konversionselektron oder Konversionspositron (gelb) und zwei unterschiedlichen Konversionselektronen oder Konversionspositronen (grau).}
\label{fig:SigTemp}
\end{figure}
\newline
%Joshuas Problem mit dem Satz?
Abbildung \ref{fig:SigTemp} zeigt das Template des Signals in grün, sowie die Aufteilung des Signals in seine einzelnen Komponenten.
%wording anders?
Die Komponenten setzen sich aus den drei möglichen Kombinationen von \textit{Clustern} zusammen.
Zum einen aus \textit{Clustern} aus Photonen, in der Abbildung als $\gamma$ bezeichnet, und zum anderen aus \textit{Clustern} aus einem Elektron oder Positron, die durch die Konversion  eines Photonen entstanden sind.
Letztere werden in der Abbildung durch $\gamma_\text{conv}$ symbolisiert.
\newline
In blau sind die Kombinationen aus zwei Photonen ($\gamma\gamma$) dargestellt, in gelb die Kombination aus Photon und Elektron oder Positron ($\gamma\gamma_\text{conv}$) und in grau die Kombination aus Konversionselektron oder Konversionspositron miteinander ($\gamma_\text{conv}\gamma_\text{conv}$).
\newline
Die Abbildung zeigt außerdem, wie zuvor angesprochen, dass bis $m_\text{inv}>0\,05 \text{ GeV}/c^{2}$ Signal vorliegt.
Der Anteil des Signals um diese invariante Masse besteht hauptsächlich aus zwei Teilchen einer Photonkonversion.
Genau dieser Teil des Signals wird nicht durch die Standardmethode berücksichtigt.
Durch das Berücksichtigen in der Analyse mit Hilfe der Templates kann ein größerer Anteil des Signals gezählt werden.
Deshalb wird eine geringere statistische Unsicherheit erwartet.