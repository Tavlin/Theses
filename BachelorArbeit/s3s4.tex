Signal == statistische Groesse


Wie im Abschnitt zuvor erw\"ahnt, wird in diesem Abschnitt die Extraktion des Signals, beziehungsweise die Absch\"atzung des korrelierten Untergrunds durch parametrisieren von Funktionen kurz vorgestellt.
\newline
Da es sich bei dem Signal um eine statistische Gr\"o{\ss}e handelt, wird eine gau{\ss}f\"ormig Funktion benutzt, um das Signal zu beschreiben.
Aufgrund der bereits angesprochenen Kombination zweier Photonenkandidaten, bei denen mindestens einer der beiden Photonenkandidaten ein Elektron oder Positron ist, die aus dem gleichen $\pi^{0}$ kommen, aber eine geringere invariante Masse besitzen, wird die gau{\ss}f\"ormig Funktion um eine sogenannte \textit{Tail} Komponente erweitert.
Die \textit{Tail} Komponente wird durch eine exponentielle Funktion beschrieben, die anschaulich als eine Abweichung der gau{\ss}f\"ormig Funktion des Signals auf der linken Seite betrachtet werden kann.
