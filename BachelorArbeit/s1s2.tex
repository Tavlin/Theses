Farbladung hat drei m\"ogliche \glqq{}Werte\grqq{}: rot, blau und gr\"un.
Die drei Antifarben sind entsprechend antirot, antiblau und antigr\"un.
Die Kombination der drei (Anti)Farben, oder die Kombination Farbe mit passender Antifarbe ergibt wei{\ss}.
Quarks und Gluonen tragen jeweils Farbladung.
Dadurch sind sowohl Quarks auch auch Gluonen in der Lage an der starken Wechselwirkung teilzunehmen.
Die starke Wechselwirkung ist für die Bindung von Quarks und Gluonen verantwortlich.
Teilchen, die aus Quarks aufgebaut sind, kann man unterteilen in sogenannte Baryonen $(qqq)$ und Mesonen ($q\bar{q}$), sowie entsprechend Antiteilchen. Alle aus Quarks bestehenden Teilchen nennt man Hadronen.
Proton und Neutronen sind beispielsweise zwei Hadronen, welche aus $u$ und $d$ bestehen.
Die Gluonen sind dabei als virtuelle Teilchen f\"ur einen st\"andigen Farbaustausch innerhalb von Hadronen verantwortlich.
Hadronen selbst sind dabei immer wei{\ss}, man sagt auch, Hadronen sind farbneutral nach au{\ss}en hin.
In der Natur kommen nur farbneutrale Teilchen vor, es gibt keine freie Farbladung.
Das Ph\"anomen, das es keine freien farbgeladenen Teilchen gibt ist das sogenannte \textit{Confinement}.
Um das \textit{Confinement} besser zu verstehen muss man sich die Kraft, beziehungsweise das Potential, der starken Wechselwirkung genauer ansehen.

Die Kraft, die auf farbgeladene Teilchen wirkt, folgt aus einem Potential $V(r)$.
$V(r)$ besitzt einen anziehenden Teil und einen abstoßenden Teil.
Der anziehende Teil ist dabei proportional zum Abstand $r$ zweier farbgeladener Teilchen, w\"ahrend der absto{\ss}ende Teil antiproportional zu $r$ ist.
Der absto{\ss}ende Teil ist zus\"atzlich proportional zur Kopplungskonstante der starken Wechselwirkung $\alpha_\text{s}$.
Es gilt:
\begin{align} \label{eq:Potential}
V(r) = -\frac{4}{3}\frac{\alpha_\text{s}}{r} + kr
\end{align}
F\"ur gro{\ss}e $r$ wird der anziehende Teil also immer stärker.
Will man also zwei farbgeladene Teilchen wie etwa ein Quark-Antiquark-Paar von einander trenne, so m\"usste man immer mehr Energie aufwenden, je weiter sich die beiden Teilchen von einander weg befinden.
Ab einem bestimmen Punkt wird die ben\"otigte Energie so gro{\ss}, dass sie ausreicht ein weiters Quark-Antiquark-Paar zu erzeugen.
Deshalb sind Quarks und Gluonen nicht direkt messbar, was die Untersuchung von Quarks, Gluonen und der starken Wechselwirkung kompliziert macht.
Um zu erkl\"aren, wie die starke Wechselwirkung, Quarks und Gluonen trotzdem untersuchen werden k\"onnen muss man sich $\alpha_\text{s}$ genauer anschauen. 

Anders als die Bezeichnung vermuten l\"asst ist die Kopplungskonstante n\"amlich nicht konstant.
Stattdessen h\"angt $\alpha_\text{s}$ vom Impulsübertragsquadrat $Q^{2}$ zwischen zwei Teilchen ab.
Das Impulsübertragsquadrat $Q^{2}$, bzw. der Impulsübertrag $Q$ h\"angt dabei selbst \"uber die De-Broglie-Wellenl\"ange mit dem Abstand $r$ zusammen.
Es gilt $Q = \frac{h}{\lambda}$, wobei $\lambda$ die r\"aumliche Aufl\"osung beschreibt.
F\"ur eine genau Aufl\"osung, also f\"ur  sehr kleine $r$ muss $Q$ und damit auch $Q^{2}$ gro{\ss} sein.
$\alpha_\text{s}$ h\"angt also antiproportional von $r$ ab.
Aufgrund dieses Zusammenhangs nennt man $\alpha_\text{s}$ auch \textit{running $\alpha_\text{s}$}. Abbildung \ref{FEHLT} [BILD] zeigt den Verlauf von $\alpha_\text{s}$ in Abh\"ahngigkeit von $Q^{2}$.
Den Zustand f\"ur sehr kleine $\alpha_\text{s}$ nennt man asymptotische Freiheit, da sich innerhalb dieses Zustands Quarks und Gluonen quasi frei bewegen k\"onnen.
Um so einen Zustand erzeugen zu k\"onnen braucht man eine hohe Dichte von Quarks und Gluonen oder eine hohe Temperatur.
Eine verbreitete theoretische Beschreibung eines Mediums in diesem hei{\ss}en und dichten Zustand ist das sogenannte Quark-Gluon-Plasma.

Ein hei{\ss}er und dichter Zustand wird bei der Kollision von zwei Atomkernen erzeugt.
Dieser Zustand entsteht kurz nach der Kollision.
Quarks und Gluonen, die aus diesem Medium kommen, m\"ussen, in der sogenannten Hadronisierung, wieder zu Hadronen werden. Diese Hadronen zerfallen, insofern sie keine stabilen Teilchen sind, wie das Proton.
Es kann auch zu ganzen Zerfallsketten kommen, bis die Endteilchen nicht mehr zerfallen
Diese Endteilchen k\"onnen gemessen werden und liefern indirekt Aufschluss auf Eigenschaften des hei{\ss}en und dichten Zustands.