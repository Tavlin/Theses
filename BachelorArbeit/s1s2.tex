Die starke Wechselwirkung ist für die Bindung von Quarks verantwortlich. Durch die starke Wechselwirkung gebundene Zust\"ande hei{\ss}en hadronen. Dabei gibt es zwei M\"oglichkeiten, wie sich Quarks binden k\"onnen. Zum einen bilden drei Quarks ($qqq$) ein Baryon und ein Quark mit einem Antiquark ($q\bar{q}$) ein Meson. Neben Baryonen gibt es auch entsprechende Antiteilchen, welche aus drei Antiquarks bestehen ($\bar{q}\bar{q}\bar{q}$).

Die zwei bekanntesten Baryonen hei{\ss}en Proton ($uud$) und Neutron ($udd$). Protonen und Neutronen die den Atomkern bilden sind also anders als die Elektron, die um den Atomkern kreisen, nicht elementar.

Wie bereits zuvor angesprochen tragen Quarks eine gewisse Farbladung. Die g\"annigste Bennenung der drei (Anti-)Farben ist rot, blau und gr\"un, in Anlehnung an die Farblehre, da eine Kombination aller drei Farben wei{\ss} ergibt. Ebenso ergibt Farbe und entsprechende Antifarbe auch wei{\ss}.
Mesonen oder Baryonen sind jedoch farbneutral nach au{\ss}en hin.
Anders als in der elektromagnetischen Wechselwirkung die Photonen, tragen die Gluonen als Austauschteilchen ebenfalls Farbladung und können somit auch selbst wechselwirken.

Die Kraft die auf farbgeladene Teilchen wirkt folgt aus einem Potential $V(r)$. F\"ur dieses Potential gilt:
\begin{align} \label{eq:Potential}
V(r) = -\frac{4}{3}\frac{\alpha_\text{s}}{r} + kr
\end{align}
$\alpha_\text{s}$ bezeichnet die Kopplungskonstante der starken Wechselwirkung. Der anziehende Teil des Potentials ist also proportional zum Abstand der beiden farbgeladenen Teilchen, w\"ahrend der absto{\ss}ende Teil mit zunehmendem Abstand kleiner wird. Anders als in der QED wird die Anziehung zweier Teilchen also immer stärker je weiter man sie von einander entfernt. Die Feldlinien verdichten sich und durch die zuvor angesprochene M\"oglichkeit von Gluonen miteinander zu wechselwirken, verbinden sich die Feldlinien. Eine solche Kette von Feldlinien bezeichent man auch als \textit{string}. 

Wenn man zwei Teilchen weit genug von einander entfernt ist die benötigte Energie um sie weiter voneinander zu entfernen so gro{\ss}, dass es zum sogenannten \textit{stringbreaking} kommt und ein neues Teilchen Antiteilchen Paar entsteht. Dies ist möglich nach der \"Aquivalenz von Masse und Energie wie sie 1905 von Albert Einstein entdeckt wurde.

Das hat zur Folge, dass es in der Natur keine freien Teilchen mit Farbladung gibt. Dieses Verhalten nennt man \textit{confinement}.

Um die starke Wechselwirkung und die Quarks und Gluonen untersuchen zu k\"onnen muss dieses confinement aufgebrochen werden. Dabei spielt die Kopplungskonstante $\alpha_\text{s}$ eine Rolle. Anders als die Bezeichnung vermuten l\"asst ist sie n\"amlich nicht konstant. Stattdessen h\"angt $\alpha_\text{s}$ vom Impulsübertragsquadrat $Q^{2}$ zwischen zwei Teilchen ab. Aufgrund dieses Zusammenhangs nennt man die Wechselwirkung auch \textit{running coupling}. 

Das Impulsübertragsquadrat $Q^{2}$, bzw. der Impulsübertrag $Q$ h\"angt dabei selbst \"uber die De-Broglie-Wellenl\"ange mit dem Abstand $r$ zusammen. Es gilt $Q = \frac{h}{\lambda}$, wobei $\lambda$ die r\"aumliche Aufl\"osung beschreibt. F\"ur eine genau Aufl\"osung, also f\"ur  sehr kleine $r$ muss $Q$ gro{\ss} sein.
Zusammengefasst h\"angt $\alpha_\text{s}$ also von $r$ an. Dabei ist $\alpha_\text{s}(r)$ klein f\"ur kleine $r$, bzw. gro{\ss}e $Q$.
Wenn $\alpha_\text{s}$ klein genug wird k\"onnen Quarks und Gluonen theoretisch frei beobachtet werden. Dieser Verlauf von $\alpha_\text{s}$ wird Asymptotische Freiheit genannt. Einen Zustand in dem Quarks und Gluonen sich quasi frei bewegen k\"onnen nennt man Quark-Gluonen-Plasma.