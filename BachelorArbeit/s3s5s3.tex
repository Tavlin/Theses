Parametrisierungsmethode
Die Parametrisierung der beiden Templates erfolgt durch die sogenannte $\chi^{2}$-Minimierung.
$\chi^{2}$ gibt dabei als ein Ma{\ss} an wie gut eine Verteilung an gegebene Daten passt.
Je kleiner $\chi^{2}$ ist, umso besser beschreibt die Verteilung die Daten, deshalb wird $\chi^{2}$ bei der Parametrisierung minimiert.
Als freie Parameter werden zwei Skalierungsfaktoren benutzt, einmal ein Skalierungsfaktor f\"ur das Template des Signals (SF\subscript{Signal}) und einmal ein Skalierungsfaktor f\"ur das Template des korrelierten Untergrunds (SF\subscript{korr. Untergrund}).