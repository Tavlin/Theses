Neben der direkten Referenz können in der Untersuchung von pp-Kollisionen aber auch Informationen über die stark wechselwirkende Materie beziehungsweise über die starke Wechselwirkung selbst gewonnen werden.
pp-Kollisionen haben hierbei den Vorteil, dass sie besser theoretisch verstanden sind als Kern-Kern-Kollisionen.
Dabei führt man unter anderem die sogenannte Partonendichtefunktion der Protonen ein, die angibt, wie wahrscheinlich es ist, ein (Anti)Quark oder Gluon mit einem bestimmten Impulsanteil des Protons vorzufinden.
Dies wiederum ermöglicht genauere theoretische Beschreibungen von pp-Kollisionen, bei denen im engeren Sinne die Partonen, also die (Anti)Quarks und beziehungsweise oder Gluonen, miteinander stoßen.
\newline
Bei einem solchen Stoß entstehen viele neue Teilchen.
Die Produktionsrate der neuen Teilchen wird dabei als der sogenannten \textit{Yield} in Abhängigkeit vom Transversalimpuls angegeben.
Der Transversalimpuls gibt dabei den Impulsanteil an, der senkrecht zur Strahlachse eines Kollisionsexperiments liegt.
Der Transversalimpuls wird deshalb betrachtet, da die kollidierenden Teilchen bei einem solchen Experiment keinen Transversalimpuls besitzen und der gesamte Transversalimpuls der entstandenen Teilchen deshalb aus den physikalischen Prozessen während und nach der Kollision kommt.
\newline
Ein mögliches Teilchen, das in Kollisionen produziert wird, ist das neutrale Pion.
Dieses wird in dieser Arbeit analysiert und der Yield des neutralen Pions in pp Kollisionen extrahiert.