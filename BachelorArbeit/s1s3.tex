Neben der direkten Referenz k\"onnen in der Untersuchung von Proton-Proton-Kollisionen aber auch Informationen \"uber die stark wechselwirkende Materie beziehungsweise \"uber die starke Wechselwirkung selbst gewonnen werden.
Proton-Proton-Kollisionen haben hierbei den Vorteil, dass sie besser theoretisch verstanden sind als Kern-Kern-Kollisionen.
Dabei f\"uhrt man unter anderem die sogenannte Partonendichtefunktion der Protonen ein, die angibt, wie wahrscheinlich es ist, ein (Anti)Quark oder Gluon mit einem bestimmten Impulsanteil des Protons vorzufinden.
Dies wiederum erm\"oglicht genauere theoretische Beschreibungen von Proton-Proton-Kollisionen, bei denen im engeren Sinne die Partonen, also die (Anti)Quarks und beziehungsweise oder Gluonen, miteinander sto{\ss}en.
\newline
Bei einem solchen Sto{\ss} entstehen viele neue Teilchen.
Die Produktionsrate der neuen Teilchen wird dabei als der sogenannten \textit{Yield} in Abh\"angigkeit vom Transversalimpuls angegeben.
Der Transversalimpuls gibt dabei den Impulsanteil an, der senkrecht zur Strahlachse eines Kollisionsexperiments liegt.
Der Transversalimpuls wird deshalb betrachtet, da die kollidierenden Teilchen bei einem solchen Experiment keinen Transversalimpuls besitzen und der gesamte Transversalimpuls der entstandenen Teilchen deshalb aus den physikalischen Prozessen w\"ahrend und nach der Kollision kommt.
\newline
Ein m\"ogliches Teilchen, das in Kollisionen produziert wird, ist das neutrale Pion.
Dieses wird in dieser Arbeit analysiert und der Yield des neutralen Pions in Proton-Proton Kollisionen extrahiert.
%F\"ur eine detailliertere Beschreibung von Monte Carlo Simulationen sei an dieser Stelle \cite{thesis:Krissy} verwiesen.
%In dieser Arbeit werden sogenannte Templates f\"ur die Analyse von neutralen Pionen verwendet um die Produktionsrate von neutralen Pionen in Proton-Proton-Kollisionen zu bestimmen.
%Templates beschreiben in dieser Analyse Verteilungen der invarianten Masse, die aus Monte Carlo Simulationen stammen.
%Durch das Verwenden von Templates in dieser Arbeit kann das theoretische Verst\"andnis von Proton-Proton-Kollisionen \"uberpr\"uft und eventuell verbessert werden.
%Eine genauere Erl\"auterung der Templates die f\"ur diese Arbeit verwendet werden, folgt in den Abschnitten \ref{s3s5s1} und \ref{s3s5s2}.
