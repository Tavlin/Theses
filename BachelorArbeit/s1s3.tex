Wie eben erw\"ahnt k\"onnen Proton-Proton-Kollisionen als Referenzsystem f\"ur Kern-Kern-Kollisionen benutzt werden.
Neben der direkten Referenz k\"onnen \"uber Proton-Proton-Kollisionen selbst aber auch Informationen \"uber stark Wechselwirkende Materie beziehungsweise \"uber die starke Wechselwirkung gewonnen werden.
Dabei haben Proton-Proton-Kollisionen den Vorteil das sie besser theoretisch verstanden sind im Vergleich zu Kern-Kern-Kollisionen.
So gibt es unter anderem die sogenannte Partonendichtefunktion bez\"uglich Protonen, die angibt wie wahrscheinlich es ist ein (Anti)Quark oder Gluon mit einem bestimmten Impulsanteil des Protons in diesem vorzufinden.
Dies wiederum erm\"oglicht genauere Simulationen von Proton-Proton-Kollisionen, bei denen im engeren Sinne die Partonen, also die (Anti)Quarks und beziehungsweise oder Gluonen, miteinander sto{\ss}en.
F\"ur eine detailliertere Beschreibung von Monte Carlo Simulationen sei an dieser Stelle auf die Bachelorarbeit von Frau Bsc. Schmitt verwiesen \cite{thesis:Krissy}.
\newline
In dieser Arbeit werden sogenannte Templates f\"ur die Analyse von neutralen Pionen verwendet um die Produktionsrate von neutralen Pionen in Proton-Proton-Kollisionen zu bestimmen.
Die Produktionsrate wird \"uber den sogenannten \textit{Yield} in abh\"angigkeit des Transversalimpulses angegeben.
Der Transversalimpuls gibt dabei den Impulsanteil an, der nicht in Richtung Strahlachse eines Kollisionsexperiments liegt.
Dabei wird der Transversalimpuls betrachtet, da die kollidierenden Teilchen bei einem solchen Experiment keinen Transversalimpuls besitzen und der gesamte Transversalimpuls der entstandenen Teilchen deshalb aus den physikalischen Prozessen w\"ahrend und nach der Kollision kommt.
\newline
Templates bezeichnen hierbei Verteilungen der invarianten Masse, die aus Monte Carlo Simulationen stammen.
Durch das Verwenden von Templates in dieser Arbeit kann das theoretische Verst\"andnis von Proton-Proton-Kollisionen \"uberpr\"uft und eventuell verbessert werden.
Eine genauere Erl\"auterung der Templates die f\"ur diese Arbeit verwendet werden folgt in den Abschnitten \ref{s3s5s1} und \ref{s3s5s2}.
\newline
Die eben angesprochenen gr{\"o}{\ss}en invariante Masse und Transversalimpuls werden im folgenden Abschnitt im Zusammenhang der Analyse von neutralen Pionen n\"aher erkl\"art. 
