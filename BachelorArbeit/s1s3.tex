s1s3 pp Kollisionen:
Wie eben erw\"ahnt k\"onnen Proton-Proton-Kollisionen als Referenzsystem f\"ur Kern-Kern-Kollisionen benutzt werden.
Neben der direkten Referenz k\"onnen \"uber Proton-Proton-Kollisionen selbst aber auch Informationen \"uber stark Wechselwirkende Materie beziehungsweise \"uber die starke Wechselwirkung gewonnen werden.
Dabei haben Proton-Proton-Kollisionen den Vorteil das sie besser theoretisch verstanden wurden im Vergleich zu Kern-Kern-Kollisionen.
So gibt es unter anderem die sogenannte Partonendichtefunktion bez\"uglich Protonen, die angibt wie wahrscheinlich es ist ein (Anti)Quark oder Gluon mit einem bestimmten Impulsanteil in einem Proton vorzufinden.
Dies wiederum erm\"oglicht genaue Simulationen von Proton-Proton-Kollisionen, bei denen im engeren Sinne die Partonen miteinander sto{\ss}en 
In dieser Arbeit werden Daten aus Proton-Proton-Kollisionen analysiert, da 