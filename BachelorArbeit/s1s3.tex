Das neutrale Pion $\pi^{0}$ besteht aus einem Quark-Antiquark-Paar und geh\"ohrt damit zu den Mesonen.
Genauer l\"asst sich das $\pi^{0}$ als eine \"Uberlagerung zweier quantenmechanischer Zust\"ande beschreiben:
\begin{align}
\ket{\pi^{0}} = \frac{1}{\sqrt{2}}\left(\ket{u\bar{u}}-\ket{d\bar{d}}\right) \label{eq:pi0state}
\end{align}
Ein ${\pi^{0}}$ zerf{\"a}llt zu $\left( 98,823\pm0,034\right)\%$ nach einer mittleren Wegl\"ange von ${\it c\tau} = 25,5$nm \cite{book:pdg} in zwei Photonen.
Die beiden Photonen k\"onnen detektiert werden, wodurch ihre Energie und ihre Position bekannt wird.
Durch die Information \"uber die Position der Photonen kann auch der Zerfallswinkel $\theta_{\gamma\gamma}$ bestimmt werden.
Die Energien $E_{\gamma1}$ und $E_{\gamma2}$ der beiden Photonen, sowie der Zerfallswinkel $\theta_{\gamma\gamma}$ werden ben\"otigt um die invariante Masse $m_{\text{inv}}$ eines $\pi^{0}$ zu berechnen.
Die Zahlen in den Indizes beziehen sich dabei auf die Nummerierung der beiden Photonen. Die Indizes x und y beziehen sich auf die Raumrichtungen. F{\"u}r diese gilt:
\begin{align}
m_{\text{inv}} &= \sqrt{2E_{\gamma\it{1}}E_{\gamma\it{2}}(1-\cos\left( \theta_{\gamma\gamma}\right) )} \label{eq_invmass}
\end{align}
Die Bedeutung der invarianten Masse wird in Abschnitt \ref{s3s2} verdeutlicht.
\newline
Neben der invarianten Masse wird die Aufteilung des Impulses der Photonen noch bestimmt, die wiederum notwendig ist, um den sogenannten Transversalimpuls $p_\text{T}$ des $\pi^{0}$ zu Berechnen. Es gilt:
\begin{align}
p_{T\pi^{0}} &= \sqrt{\left(p_{x1}+p_{x2}\right)^{2} +\left(p_{y1}+p_{y2}\right)^{2}} \label{eq_pt}
\end{align}
Bei einem Kollisionsexperiment besitzen die beiden kollidierenden Teilchen keinen Transversalimpuls, daher wird oft von Teilchen, die aus der Kollision kommen, nur der Transversalimpuls betrachtet.
\newline
Die Messung von $\pi^{0}$ wird aus mehreren Gr\"unden zur Untersuchung des QGP verwendet.
Zum einen um die Anzahl direkter Photonen bestimmen zu k\"onnen.
Als direkte Photonen werden solche Photonen bezeichnet, welche in der Kollision entstehen und nicht aus Zerf\"allen stammen.
Um die Anzahl direkter Photonen bestimmen zu k\"onnen wird die Anzahl indirekter Photonen, wie etwa Photonen aus einem $\pi^{0}$-Zerfall, von der Gesamtzahl aller gemessenen Photonen abgezogen.
Da $\pi^{0}$ mit am h\"aufigsten in Kollisionen produziert werden, spielt die Anzahl an Photonen aus $\pi^{0}$-Zerf\"allen eine wichtige Rolle in der Bestimmung direkter Photonen.
Direkte Photonen wiederum k\"onnen benutzt werden um etwa die Temperatur des erzeugten Mediums bestimmen.
\newline
Zum anderen wird die Anzahl produzierter $\pi^{0}$ von Kollisionen in den erwartet wird ein QGP zu erzeugen verglichen mit Kollisionen bei denen davon ausgegangen wird, dass dort kein QGP entsteht, wie etwa Proton-Proton Kollisionen.
Das Ver\"altnis der Produktionsraten abh\"angig vom Transversalimpuls kann so zum Beispiel Aufschluss geben auf den Energieverlust von Teilchen innerhalb des stark wechselwirkenden Mediums. 