In dieser Arbeit werden Messdaten des ALICE Experiments verwendet.
Das ALICE Experiment befindet sich am LHC, dem weltweit größten Beschleunigerring, am Kernforschungszentrum CERN.
Im LHC werden Teilchen, hauptsächlich Blei-Ionen und Protonen, auf fast Lichtgeschwindigkeit beschleunigt und zum Kollidieren gebracht.
Die Beschleunigung geschieht durch elektrische Felder, während Dipolmagnete die beschleunigten Teilchen auf einer Kreisbahn halten.
Kollisionen finden im LHC Ring an vier unterschiedlichen Stellen statt, wo sich die Strahlrohre, in denen Teilchen gegenläufig beschleunigt werden, kreuzen.
An einem dieser Punkte befindet sich das ALICE Experiment.
\newline
Die Beschleunigung auf nahezu Lichtgeschwindigkeit ermöglicht es, hohe Schwerpunktsenergien $\sqrt{s}$ zu erreichen.
Dabei spiegelt $\sqrt{s}$ die Energie wider, die das System in einer Kollision zur Verfügung hat.
Dementsprechend können mehr und auch schwerere Teilchen bei höherem $\sqrt{s}$  in einer Kollision entstehen.
Ein hohes $\sqrt{s}$ hat auch eine höhere Temperatur des Mediums, was bei einer solchen Kollision entstehen kann, zur Folge.
So befinden sich Messungen des ALICE Experiments am LHC im Phasendiagramm stark wechselwirkender Materie, wie es in Abbildung \ref{fig:QGPPhase} skizziert ist, bei hohen Temperaturen und einer geringen Baryonendichte.
$\sqrt{s}$ hängt dabei von der Energie der kollidierenden Teilchen ab.
Für Kollisionsexperimente zweier identischer Teilchen mit gleicher Energie $E$ gilt:
\begin{align}
\sqrt{s} = 2E \label{eq:sqrts}
\end{align}
\newline
Die in dieser Arbeit verwendeten Daten stammen von pp-Kollisionen bei $\sqrt{s} = 13$ TeV.