In Abbildung \ref{fig:QGPPhase} sind zus\"atzlich verschiedene Datenpunkte eingezeichnet, die verschiedene sogenannte Schwerpunktsenergieen $\sqrt{s}$ widerspiegeln.
Die Schwerpunktsenergie eines Kollisionsexperiments gibt an, wie viel Energie dem System bei der Kollision zur Verf\"ugung steht.
Entsprechend h\"angt $\sqrt{s}$ von der Energie der kollidierende Teilchen oder Kerne ab.
F\"ur Kollisionsexperimente zweier identischer Teilchen oder Kerne mit gleicher Energie $E$ gilt:
\begin{align}
\sqrt{s} = 2E \label{eq:sqrts}
\end{align}
Unterschiedliche $\sqrt{s}$ erlauben es unterschiedlichen Bereichen des Phasendiagramms zu studieren.
Um die Bereiche des Phasendiagramms innerhalb des QGP und dem \"Ubergang zwischen quasi freien zu gebundenen Quarks und Gluonen untersuchen zu k\"onnen, werden also Kollisionen mit ausreichenden Schwerpunktsenergieen ben\"otigt.
\newline
Um die Scherpunktsenergieen, die f\"ur die Entstehung des QGP n\"otig sind, erreichen zu k\"onnen, m\"ussen Kerne auf fast Lichtgeschwindigkeit beschleunigt werden.
Die Beschleunigung geschieht in Beschleunigerringen, wo Teilchen oder Kerne durch Dipolmagnete auf einer Kreisbahn gehalten und durch elektrische Felder beschleunigt werden.
Der LHC am Kernforschungszentrum CERN, der weltweit gr\"o{\ss}te Beschleunigerring, erreicht aktuell Schwerpunktsenergieen bis $\sqrt{s} = 13$ TeV.
Im LHC Ring kreuzen sich and vier Stellen die Strahlrohre, wo es zu Kollisionen kommen kann.
An jeder dieser vier Stellen befindet sich ein Experiment, wie etwa das ALICE Experiment.
Im folgenden Abschnitt wird das ALICE Experiment genauer beschrieben.