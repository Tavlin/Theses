Um die ben\"otigten Scherpunktsenergieen erreichen zu k\"onnen, m\"ussen die Teilchen beziehungsweise Kerne auf fast Lichtgeschwindigkeit beschleunigt werden.
Die Beschleunigung geschieht in Beschleunigerringen, wo Teilchen oder Kerne durch Dipolmagnete auf einer Kreisbahn gehalten und durch elektrische Felder beschleunigt werden.
Der LHC, der weltweit gr\"o{\ss}te Beschleunigerring, geh\"ohrt zu CERN und erreicht aktuell Schwerpunktsenergieen bis $\sqrt{s} = 13 TeV$.
Im LHC Ring befinden sich vier Punkte an denen Kollisionen stattfinden.
An diesen vier befinden sich umfangreiche Detektorkomplexe, wie etwa des ALICE Experiments.
Im folgenden Abschnitt wird das ALICE Experiment genauer beschrieben.