In dieser Arbeit werden Messdaten des ALICE Experiments verwendet.
Das ALICE Experiment befindet sich am LHC, den weltweit gr\"o{\ss}te Beschleunigerring, am Kernforschungszentrum CERN.
Im LHC werden Teilchen, haupts\"achlich Blei-Ionen und Protonen auf fast Lichtgeschwindigkeit beschleunigt und zum Kollidieren gebracht.
Die Beschleunigung geschieht durch elektrische Felder, w\"ahrend Dipolmagnete die beschleunigten Teilchen auf einer Kreisbahn halten.
Kollisionen finden im LHC Ring an vier unterschiedlichen Stellen statt, wo sich die Strahlrohre, in denen Teilchen gegenl\"aufig beschleunigt werden, kreuzen.
An einem dieser Punkte befindet sich das ALICE Experiment.
\newline
Die Beschleunigung auf nahezu Lichtgeschwindigkeit erm\"oglicht hohe Scherpunktsenergieen $\sqrt{s}$ zu erreichen.
Dabei spiegelt $\sqrt{s}$ die Energie wieder die das System in einer Kollision zur Verf\"ugung hat.
Dementsprechend k\"onnen mehr und auch schwerere Teilchen bei h\"oherem $\sqrt{s}$  in einer Kollision entstehen.
Ein hohes $\sqrt{s}$ hat auch eine h\"ohere Temperatur des Mediums was bei einer solchen Kollision entstehen kann zur Folge.
So befinden sich Messungen des ALICE Experiments am LHC im Phasendiagramm stark wechselwirkender Materie, wie es in Abbildung \ref{fig:QGPPhase} skizziert ist, bei hohen Temperaturen und einer geringen Baryonendichte.
$\sqrt{s}$ h\"angt dabei von der Energie der kollidierende Teilchen ab.
F\"ur Kollisionsexperimente zweier identischer Teilchen mit gleicher Energie $E$ gilt:
\begin{align}
\sqrt{s} = 2E \label{eq:sqrts}
\end{align}
\newline
Im folgenden Abschnitt wird das ALICE Experiment genauer beschrieben.