Wie im vorherigen Abschnitt beschrieben, ist zur Erzeugung eines hei{\ss}en und dichten Zustands die Kollision zweier Atomkerne n\"otig.
Solche Kollisionen werden in sogenannten Colliderexperimenten durchgef\"uhrt.
Ein solches Colliderexperiment wird am CERN betrieben.
Das CERN ist Europas gr\"o{\ss}tes Institut zur Untersuchung im Bereich der (Elementar)Teilchenphysik.
Am CERN befindet sich das nach aktuellem Stand gr\"o{\ss}te Colliderexperiment weltweit.
Um Kerne und andere Teilchen zum Kollidieren zu bringen m\"ussen diese zun\"achst beschleunigt werden.
Dies geschieht in mehreren sogenannten Beschleunigerringen.
Der gr\"o{\ss}te Beschleunigerring ist der LHC, in welchem die beschleunigten Teile kollidieren.
Die Kollisionen werden an vier Punkten im LHC provoziert.
An allen vier Stellen befinden sich umfangreiche Detektorkomplexe, wie etwa der ALICE Detektor.
Im folgenden Abschnitt wird das ALICE Experiment genauer beschrieben.