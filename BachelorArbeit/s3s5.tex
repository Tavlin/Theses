Um das Signal extrahieren zu k\"onnen mit Hilfe von Templates wird, wie auch bei der Standardmethode, zun\"achst eine Absch\"atzung des korrelierten Untergrunds gemacht.
Hierf\"ur werden zwei Templates an die Daten angepasst, vergleichbar wie in der Standardmethode eine Funktion bestehend aus drei Teilen an die Daten angepasst wurde.
Ein Template des Signals wird verwendet um das gesamte $\pi^{0}$ Signal zu beschreiben.
Im Vergleich zur Standardmethode entspricht das dem Gau{\ss}-Teil sowie dem \textit{Tail}-Tail der Funktion.
Der korrelierte Untergrund wird durch eine eigenes Template abegsch\"azt, statt durch einen lineare Funktion.
\newline
Im folgenden Abschnitt wird das Template des Signals diskutiert. 