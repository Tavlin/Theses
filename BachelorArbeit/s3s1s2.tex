An die \textit{Cluster} aus dem gewählten Datensatz werden unterschiedliche Anforderungen gestellt.
Tabelle \ref{tab:Cluster} listet die in dieser Analyse gestellten Anforderungen auf.
\begin{table}[b]
\centering
\begin{tabular}{|l|r|}
\hline
\multicolumn{2}{|c|}{Clusterauswahlkriterien}                   \\ \hline \hline
Zeit                    & $-30\text{ ns} < t_\text{clust}<35\text{ ns}$ \\ \hline
\textit{Bad Cell Map}   & angewandt                                     \\ \hline
Energie                 & $E\geq700\text{ MeV}$                         \\ \hline
Anzahl Zellen           & $N_\text{Zellen}\geq 2$                       \\ \hline
Form                    & $0.1< M_{02}<0.7$                             \\ \hline
\textit{Track matching} & angewandt                                     \\ \hline
Öffnungswinkel          & $\theta>0.017\text{ rad}$                     \\ \hline
\end{tabular}
\caption{Auswahlkriterien für die \textit{Cluster} des EMCals.}
\label{tab:Cluster}
\end{table}
\newline
Der zeitliche Rahmen, in dem die \textit{Cluster} enstehen können wird eingeschränkt, um \textit{Cluster} von Teilchen auszuschließen, die nicht von einem \textit{minimum-bias event} stammen.
Die \textit{Bad Cell Map} wird verwendet um schlechte Zellen von der Analyse auszuschließen \cite{thesis:Joshua}.
\newline
Die Energie die ein \textit{Cluster} mindestens braucht, sowie die Anforderung an die Form und die Anzahl an Zellen, aus denen ein \textit{Cluster} mindestens bestehen muss, helfen \textit{Cluster} von Photonen, und Konversionselektronen beziehungsweise Konversionspositronen zu selektieren.
Die Mindestenergie steht dabei im direkten Zusammenhang mit der Mindestanzahl an Zellen.
Wie Abschnitt \ref{s2s2} erwähnt, benötigt die Startzelle mindestens 600 MeV und jede weitere Zelle mindestens 100 MeV, um zu einem \textit{Cluster} hinzugefügt zu werden.
Die Form angegeben durch den Parameter $M_{02}$ wurde in Abschnitt \ref{s2s2} ebenfalls bereits erläutert.
Mit Hilfe des \textit{Track matching} können \textit{Cluster}, die von geladenen Teilchen kommen, ausgeschlossen werden, indem Spuren der geladenen Teilchen, die in der TPC hinterlassen wurden, rekonstruiert und bis zum EMCal extrapoliert werden.
\newline
Durch die Anforderung an den Öffnungswinkel wird sicher gestellt, dass der Abstand zwischen zwei \textit{Clustern} mindestens der Größe einer Zelle entspricht.
Dementsprechend werden zwei Teilchen, die sehr nah nebeneinander auf das EMCal treffen, zu einem \textit{Cluster} zusammengefasst.
\newline
