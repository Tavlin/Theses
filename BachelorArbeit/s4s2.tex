Systematische Unsicherheiten:
Für den korrigierten Yield wird zu Letzt noch die systematische Unsicherheit bestimmt.
Dabei wird sich in dieser Arbeit rein auf die systematische Unsicherheit, die durch Variation in der Peakeextraktion kommt, fokussiert.
Die Variationen die in dieser Arbeit verwendet wurden lassen sich in vier Abschnitte unterteilen.
Bei der \textbf{Variation des Zählbereiches}, wird der Zählbereich einmal ausgeweitet und ein anderes mal verkleinert.
Für das Verkleinern wird dabei der untere Wert für $m_\text{inv}$ um $ZAHL$ erhöht, während der obere Wert für $m_\text{inv}$ um $ZAHL$ verringert wird.
Entsprechend wird beim Ausweiten der untere Wert für $m_\text{inv}$ um $ZAHL$ verringert der obere Wert für $m_\text{inv}$ um $ZAHL$ erhöht.
Bei der \textbf{Variation des Parametrisierungsbereiches} wird analog verfahren, mit den gleichen Zahlenwerten.
\newline
