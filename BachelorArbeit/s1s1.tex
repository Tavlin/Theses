Die Kraft, die zwischen zwei Teilchen wirkt, wird als Wechselwirkung (kurz: WW) bezeichnet. Nach dem heutigen Stand der Wissenschaft gibt es vier fundamentale WW: die schwache WW, die starke WW, die elektromagnetische WW und die Gravitation. Eine Theorie, das so genannte Standardmodell der Elementarteilchenphysik, vereint alle WW bis auf die Gravitation und beschreibt die Physik der elementarsten Teilchen.

WW zwischen zwei Teilchen basiert auf der Ladung die die beiden Teilchen tragen. 
Elementarteilchen sind der Grundbaustein von Materie und zeichnen sich dadurch aus, dass sie nach aktuellem Wissensstand nicht weiter teilbar sind.

Diese Elementarteilchen wechselwirken miteinander aufgrund ihrer Ladung(en). Das Elektron z.B. hat eine elektrische Ladung von $-1$. Durch die elektrische Ladung ist es in der Lage an elektromagnetischer Wechselwirkung teilzunehmen. Wenn zwei Teilchen miteinander wechselwirken, dann geschieht das unter Austausch eines sogenannten Austauschteilchens, im Falle der elektromagnetischen Wechselwirkung durch Austausch eines Photons.

Die Elementarteilchen kann man nun entsprechend ihrer elektrischen Ladung und der Generation unterteilen. Ebenso lassen sich die Austauschteilchen ihrer entsprechenden Wechselwirkung zuordnen, wobei diese ebenfalls zu den Elementarteilchen z{\"a}hlen. Dies wurde in den beiden Tabellen \ref{tab:teilchen} und \ref{tab:Austeilchen} gemacht. Es gibt also sechs Quarks und sechs Leptonen, mit jeweils ihren entsprechend Antiteilchen.

\begin{table}[h] 
\centering
\begin{tabular}{|c||c|c|c||c|}
\hline
Generation & I                                                                                    & II                                                                              & III                                                                             & el. Ladung                                           \\ \hline \hline
Quarks    & \begin{tabular}[c]{@{}c@{}}up ($u$)\\ down ($d$)\end{tabular}                        & \begin{tabular}[c]{@{}c@{}}charm ($c$)\\ strange ($s$)\end{tabular}             & \begin{tabular}[c]{@{}c@{}}top ($t$)\\ bottom ($b$)\end{tabular}                & \begin{tabular}[c]{@{}c@{}}+2/3\\  -1/3\end{tabular} \\ \hline
Leptonen  & \begin{tabular}[c]{@{}c@{}}Elektron ($e$)\\ Elektron-Neutrino ($\nu_{e}$)\end{tabular} & \begin{tabular}[c]{@{}c@{}}Myon($\mu$)\\ Myon-Neutrino ($\nu_{\mu}$)\end{tabular} & \begin{tabular}[c]{@{}c@{}}Tau($\tau$)\\ Tau-Neutrino ($\nu_{\tau}$)\end{tabular} & \begin{tabular}[c]{@{}c@{}}-1\\  0\end{tabular}      \\ \hline
\end{tabular}
\caption{Tabelle der Elementarteilchen}
\label{tab:teilchen}
\end{table}


\begin{table}[h]
\centering
\begin{tabular}{|c||c|c|c|}
\hline
Wechselwirkung    & elektromagnetisch & stark       & schwach                      \\ \hline
Austauschteilchen & Photon ($\gamma$) & Gluon ($g$) & $W^{\pm}$, $Z^{0}$ - Bosonen \\ \hline
\end{tabular}
\caption{Tabelle der Austauschteilchen}
\label{tab:Austeilchen}
\end{table}

Zus\"atzlich zu den bisher genannten Gr\"o{\ss}en lassen sich den Elementarteilchen noch weitere Eigenschaften zuschreiben. So haben alle drei Austauschteilchen einen Spin von $1$ und geh\"ohren somit zu den Bosonen. Quarks und Leptonen tragen hingegen einen Spin von $1/2$ und geh\"ohren entsprechend zu den Fermionen.

Wie oben bereits erw\"ahnt koppeln Kr\"afte an die Ladung von Teilchen. Um die elektromagnetische Kraft zu Beschreiben gibt es die Quantenelektrodynamik (QED). F\"ur die starke Wechselwirkung gibt es entsprechend die Quantenchromodynamik (QCD). Der Name kommt daher, dass die Ladung, die zur starken Wechselwirkung geh\"ohrt, die Farbladung ist (chroma ist griechisch für Farbe). Die Farbladung hat dabei nichts mit der \"au{\ss}eren Erscheinung der Quarks zu tun.