Das Standardmodell der Elementarteilchenphysik unterteilt die sogenannten Elementarteilchen in zwei Gruppen, die sogenannten Quarks und die sogenannten Leptonen.
Als Elementarteilchen werden alle Teilchen bezeichnet, die nicht weiter teilbar sind.
Beide Gruppe beinhalten nach aktuellem Wissensstand jeweils sechs Teilchen, die sechs Quarks, \textit{up} ($u$), \textit{down} ($d$), \textit{strange} ($s$), \textit{charm} ($c$), \textit{bottom} ($b$) und \textit{top} ($t$) und die sechs Leptonen, Elektron ($e$), Elektron-Neutrino ($\nu_\text{e}$), Myon ($\mu$), Myon-Neutrino ($\nu_{\mu}$), Tau ($\tau$) und Tau-Neutrino ($\nu_{\tau}$).
In Tabelle \ref{tab:teilchen} sind die Elementarteilchen bez\"uglich ihrer sogenannten Generation und ihrer elektrischen Ladung sortiert aufgelistet.
Die Aufteilung in die Generationen erfolgt nach der Masse der Elementarteilchen, so besteht die erste Generation aus den leichtesten Elementarteilchen, die dritte Generation hingegen aus den schwersten Elementarteilchen.
Die Generationen der Leptonen und Quarks sind dabei unabh\"ahning voneinander.
Zu jedem Elementarteilchen existiert ein  entsprechendes Antiteilchen, mit entgegengesetzten Ladungen.

\begin{table}[h] 
\centering
\begin{tabular}{|c||c|c|c||c|}
\hline
Generation & I                                                                                    & II                                                                              & III                                                                             & el. Ladung [e]                                          \\ \hline \hline
Quarks    & \begin{tabular}[c]{@{}c@{}}up ($u$)\\ down ($d$)\end{tabular}                        & \begin{tabular}[c]{@{}c@{}}charm ($c$)\\ strange ($s$)\end{tabular}             & \begin{tabular}[c]{@{}c@{}}top ($t$)\\ bottom ($b$)\end{tabular}                & \begin{tabular}[c]{@{}c@{}}+2/3\\  -1/3\end{tabular} \\ \hline
Leptonen  & \begin{tabular}[c]{@{}c@{}}Elektron ($e$)\\ Elektron-Neutrino ($\nu_{e}$)\end{tabular} & \begin{tabular}[c]{@{}c@{}}Myon($\mu$)\\ Myon-Neutrino ($\nu_{\mu}$)\end{tabular} & \begin{tabular}[c]{@{}c@{}}Tau($\tau$)\\ Tau-Neutrino ($\nu_{\tau}$)\end{tabular} & \begin{tabular}[c]{@{}c@{}}-1\\  0\end{tabular}      \\ \hline
\end{tabular}
\caption{Elementarteilchen geordnet nach ihrer Generation und ihrer elektrische Ladung}
\label{tab:teilchen}
\end{table}

Im Standardmodell der Elementarteilchenphysik wird allgemein zwischen drei Ladungen unterschieden, der elektrische Ladung, der starke Ladung und der schwache Ladung.
Jede dieser Ladungen l\"asst sich einer sogenannten Wechselwirkung zuordnen.
Die drei Wechselwirkungen innerhalb des Standardmodells sind die elektromagnetische Wechselwirkung, die starke Wechselwirkung und die schwache Wechselwirkung.
Durch das Tragen einer Ladung wird ein Teilchen durch die entsprechende Wechselwirkung beeinflusst.
Die Kraft, die ein geladenes Teilchen auf ein anderes, gleichartig geladenes Teilchen aus\"ubt resultiert aus der passenden Wechselwirkungen.
Ein Teilchen kann dabei auch mehrere der drei unterschiedlichen Ladungen tragen und somit an mehreren Wechselwirkungen teilnehmen.
Aus Wechselwirkungen resultieren, neben der Kraft von einem Teilchen auf ein anders Teilchen, aber auch Zerf\"alle oder Annihilationen von Teilchen.
\newline
Kr\"afte zwischen zwei Teilchen werden durch das Austauschen von einem sogenannten Austauschteilchen vermittelt.
%Innerhalb eines solchen Austauschs sind Austauschteilchen virtuelle Teilchen und k\"onnen deshalb nicht gemessen werden.
Zu den bekannten Austauschteilchen geh\"oren das Photon ($\gamma$), das Gluon ($g$), das Z-Boson und die W-Bosonen ($Z^{0}$ \& $W^{\pm}$).
Tabelle \ref{tab:Austeilchen} zeigt die Zuordnung der Austauschteilchen zu ihrer entsprechende Wechselwirkung.
%Das Photon und die Gluonen sind dabei f\"ur diese Arbeit wichtig.
Im folgenden Abschnitt wird genauer auf Quarks, Gluonen und die Ladung der starken Wechselwirkung eingegangen.

\begin{table}[h]
\centering
\begin{tabular}{|c||c|c|c|}
\hline
Wechselwirkung    & elektromagnetisch & stark       & schwach                      \\ \hline
Austauschteilchen & Photon ($\gamma$) & Gluon ($g$) & $W^{\pm}$, $Z^{0}$ - Bosonen \\ \hline
\end{tabular}
\caption{Austauschteilchen der entsprechende Wechselwirkung zugeordnet}
\label{tab:Austeilchen}
\end{table}
