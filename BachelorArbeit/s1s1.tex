Die Kraft, die ein Teilchen auf ein anderes auswirkt, wird durch Wechselwirkung vermittelt. Man kann auch sagen, dass die Kraft, beziehungsweise die Wechselwirkung an ein Teilchen koppelt. Die st\"arke der Kopplung spiegelt sich in der sogenannten Kopplungskonstante wieder.  Wechselwirkungen umfassen, neben der Kraft von einem Teilchen auf ein anders, aber auch Zerf\"alle oder Annihilationen. Nach dem heutigen Stand der Wissenschaft gibt es vier fundamentale Wechselwirkung: die schwache Wechselwirkung, die starke Wechselwirkung, die elektromagnetische Wechselwirkung und die Gravitation. Das Standardmodell der Elementarteilchenphysik beschreibt als eine Theorie die starke, die schwache und die elektromagnetische Wechselwirkung, sowie alle bekannten Elementarteilchen.
Elementarteilchen sind der Grundbaustein von Materie und zeichnen sich dadurch aus, dass sie, nach aktuellem Wissensstand, nicht weiter teilbar sind.

Die Elementarteilchen teilen sich in zwei Gruppen auf, die Quarks und die Leptonen. Beide Gruppe beinhalten sechs Teilchen. Es gibt folgende sechs Quarks: \textit{up} ($u$), \textit{down} ($d$), \textit{strange} ($s$), \textit{charm} ($c$), \textit{bottom} ($b$) und \textit{top} ($t$) und folgende sechs Leptonen: Elektron ($e$), Elektron-Neutrino ($\nu_\text{e}$), Myon ($\mu$), Myon-Neutrino ($\nu_{\mu}$), Tau ($\tau$) und Tau-Neutrino ($\nu_{\tau}$). Beide Gruppen kann man weiter aufteilen in insgesamt drei Generationen. Die Generationen der Leptonen und Quarks h\"angen dabei nicht zusammen und sind mit der Masse aufsteigend geordnet. In Tabelle \ref{tab:teilchen} sind die Elementarteilchen bez\"uglich ihrer Generation und Ladung sortiert aufgelistet. Zu den Elementarteilchen gibt es entsprechende Antiteilchen, mit umgekehrter Ladung, wie beispielsweise das Positron, das Antiteilchen des Elektrons mit einfach positiver elektrischer Ladung.
F\"ur diese Arbeit sind besonders das $u$ und $d$ Quark, sowie das Elektron und die entsprechenden Antiteilchen von Bedeutung.

\begin{table}[h] 
\centering
\begin{tabular}{|c||c|c|c||c|}
\hline
Generation & I                                                                                    & II                                                                              & III                                                                             & el. Ladung                                           \\ \hline \hline
Quarks    & \begin{tabular}[c]{@{}c@{}}up ($u$)\\ down ($d$)\end{tabular}                        & \begin{tabular}[c]{@{}c@{}}charm ($c$)\\ strange ($s$)\end{tabular}             & \begin{tabular}[c]{@{}c@{}}top ($t$)\\ bottom ($b$)\end{tabular}                & \begin{tabular}[c]{@{}c@{}}+2/3\\  -1/3\end{tabular} \\ \hline
Leptonen  & \begin{tabular}[c]{@{}c@{}}Elektron ($e$)\\ Elektron-Neutrino ($\nu_{e}$)\end{tabular} & \begin{tabular}[c]{@{}c@{}}Myon($\mu$)\\ Myon-Neutrino ($\nu_{\mu}$)\end{tabular} & \begin{tabular}[c]{@{}c@{}}Tau($\tau$)\\ Tau-Neutrino ($\nu_{\tau}$)\end{tabular} & \begin{tabular}[c]{@{}c@{}}-1\\  0\end{tabular}      \\ \hline
\end{tabular}
\caption{Elementarteilchen geordnet nach ihrer Generation und ihrer elektrische Ladung}
\label{tab:teilchen}
\end{table}

Eine Eigenschaft die Teilchen besitzen k\"onnen ist ihre Ladung. Durch das Tragen einer spezifischen Ladung sind Teilchen in der Lage an einer der vier Wechselwirkungen teilzunehmen. Die H\"ohe der Ladung bestimmt, wie stark ein Teilchen durch die entsprechende Wechselwirkung beeinflusst wird. Dabei gilt es zwischen verschiedenen Ladungen zu unterscheiden. Die elektrische Ladung erm\"oglicht die Teilnahme an der elektrischen Wechselwirkung, die sogenannte Farbladung hingegen die Teilnahme an der starken Wechselwirkung. Der Vollst\"andigkeit halber sei an dieser Stelle die schwache Ladung f\"ur die schwache Wechselwirkung und die Masse f\"ur die Gravitation erw\"ahnt.

Das Koppeln von Kr\"aften zwischen zwei Teilchen wird im Standardmodell durch das Austauschen von einem sogenannten Austauschteilchen vermittelt. Innerhalb eines solchen Austauschs sind Austauschteilchen virtuelle Teilchen und k\"onnen deshalb nicht gemessen werden. Die bekannten Austauschteilchen sind Photonen ($\gamma$), Gluonen ($g$), Z- und W-Bosonen ($Z^{0}$ \& $W^{\pm}$). Alle genannten Austauschteilchen besitzen einen ganzzahligen Spin von 1 und sind deshalb Bosonen. Die Zuordnung der Austauschteilchen zu ihrer entsprechende Wechselwirkung ist in Tabelle \ref{tab:Austeilchen} zu sehen. Das Photon und die Gluonen sind dabei f\"ur diese Arbeit wichtig. Im folgenden Kapitel wird genauer auf Quarks, Gluonen und die Farbladung eingegangen.

\begin{table}[h]
\centering
\begin{tabular}{|c||c|c|c|}
\hline
Wechselwirkung    & elektromagnetisch & stark       & schwach                      \\ \hline
Austauschteilchen & Photon ($\gamma$) & Gluon ($g$) & $W^{\pm}$, $Z^{0}$ - Bosonen \\ \hline
\end{tabular}
\caption{Austauschteilchen der entsprechende Wechselwirkung zugeordnet}
\label{tab:Austeilchen}
\end{table}
