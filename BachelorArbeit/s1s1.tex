Im Standardmodell der Elementarteilchenphysik werden die sogenannten Elementarteilchen in zwei Gruppen, die sogenannten Quarks und die sogenannten Leptonen, unterteilt.
Als Elementarteilchen werden alle Teilchen bezeichnet, die nach heutigem Kenntnisstand nicht weiter teilbar sind.
Beide Gruppen beinhalten nach aktuellem Wissensstand jeweils sechs Teilchen, die sechs Quarks \textit{up} ($u$), \textit{down} ($d$), \textit{charm} ($c$), \textit{strange} ($s$), \textit{top} ($t$) und \textit{bottom} ($b$) und die sechs Leptonen Elektron ($e$), Elektron-Neutrino ($\nu_\text{e}$), Myon ($\mu$), Myon-Neutrino ($\nu_{\mu}$), Tau ($\tau$) und Tau-Neutrino ($\nu_{\tau}$).
Tabelle \ref{tab:teilchen} listet die Elementarteilchen, geordnet nach ihrer sogenannten Generation und ihrer elektrischen Ladung, auf.
%maybe weglassen?
%Die Aufteilung in die Generationen erfolgt nach der Masse der Elementarteilchen, so besteht die erste Generation aus den leichtesten Elementarteilchen, die dritte Generation hingegen aus den schwersten Elementarteilchen.
%Die Generationen der Leptonen und Quarks sind dabei unabh\"ahning voneinander.
\begin{table}[b] 
\centering
\begin{tabular}{|c||c|c|c||c|}
\hline
Generation & I                                                                                    & II                                                                              & III                                                                             & el. Ladung [e]                                          \\ \hline \hline
Quarks    & \begin{tabular}[c]{@{}c@{}}up ($u$)\\ down ($d$)\end{tabular}                        & \begin{tabular}[c]{@{}c@{}}charm ($c$)\\ strange ($s$)\end{tabular}             & \begin{tabular}[c]{@{}c@{}}top ($t$)\\ bottom ($b$)\end{tabular}                & \begin{tabular}[c]{@{}c@{}}+2/3\\  -1/3\end{tabular} \\ \hline
Leptonen  & \begin{tabular}[c]{@{}c@{}}Elektron ($e$)\\ Elektron-Neutrino ($\nu_{e}$)\end{tabular} & \begin{tabular}[c]{@{}c@{}}Myon($\mu$)\\ Myon-Neutrino ($\nu_{\mu}$)\end{tabular} & \begin{tabular}[c]{@{}c@{}}Tau($\tau$)\\ Tau-Neutrino ($\nu_{\tau}$)\end{tabular} & \begin{tabular}[c]{@{}c@{}}-1\\  0\end{tabular}      \\ \hline
\end{tabular}
\caption{Elementarteilchen geordnet nach ihrer Generation und ihrer elektrische Ladung. \cite{book:pdg}}
\label{tab:teilchen}
\end{table}
\newline
Neben der elektrische Ladung gibt es im Rahmen des Standardmodells noch zwei weitere Ladungen, die schwache Ladung und die starke Ladung, auch Farbladung genannt.
%%%% Wechselwirkung Erklären
Tr\"agt ein Teilchen eine Ladung, so koppelt das Teilchen an eine sogenannte Wechselwirkung, die beschreiben, wie Teilchen sich gegenseitig beeinflussen k\"onnen.
Jede Ladung l\"asst sich dabei einer Wechselwirkung zuordnen,
die elektrische Ladung der elektromagnetischen Wechselwirkung, die schwache Ladung der schwachen Wechselwirkung und die Farbladung der starken Wechselwirkung.
%Die drei Wechselwirkungen werden ebenfalls vom Standardmodell der Elementarteilchenphysik beschrieben.
%noetig?
%Die Kraft, die ein geladenes Teilchen auf ein anderes, gleichartig geladenes Teilchen aus\"ubt resultiert aus der passenden Wechselwirkungen.
%Ein Teilchen kann dabei auch mehrere der drei unterschiedlichen Ladungen tragen und somit an mehreren Wechselwirkungen teilnehmen.
%Aus Wechselwirkungen resultieren, neben der Kraft von einem Teilchen auf ein anders Teilchen, aber auch Zerf\"alle oder Annihilationen von Teilchen.
\newline
Wechselwirkungen zwischen zwei Teilchen werden durch den Austausch von sogenannten Austauschteilchen vermittelt.
%Innerhalb eines solchen Austauschs sind Austauschteilchen virtuelle Teilchen und k\"onnen deshalb nicht gemessen werden.
Zu den heute bekannten Austauschteilchen geh\"oren das Photon ($\gamma$), das Gluon ($g$), das Z-Boson ($Z^{0}$) und die W-Bosonen ($W^{\pm}$).
Tabelle \ref{tab:Austeilchen} zeigt die Zuordnung der Austauschteilchen zu ihrer entsprechende Wechselwirkung.
\begin{table}[b]
\centering
\begin{tabular}{|c||c|c|c|}
\hline
Wechselwirkung    & elektromagnetisch & stark       & schwach                      \\ \hline
Austauschteilchen & Photon ($\gamma$) & Gluon ($g$) & $W^{\pm}$, $Z^{0}$ - Bosonen \\ \hline
\end{tabular}
\caption{Austauschteilchen geordnet zu ihrer entsprechenden Wechselwirkung}
\label{tab:Austeilchen}
\end{table}
\newline
%SPIELT - PI
F{\"u}r die vorliegenden Arbeit spielen die starke Wechselwirkung, Quarks, Gluonen und die Farbladung eine wichtige Rolle.
Deshalb wird im folgenden Abschnitt genauer auf diese Themen eingegangen.