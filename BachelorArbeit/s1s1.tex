Die Kraft die ein Teilchen auf ein anderes auswirkt wird durch Wechselwirkung vermittelt. Wechselwirkungen k\"onnen aber auch Zerf\"alle oder Annihilationen ausl\"osen. Nach dem heutigen Stand der Wissenschaft gibt es vier fundamentale Wechselwirkung: die schwache Wechselwirkung, die starke Wechselwirkung, die elektromagnetische Wechselwirkung und die Gravitation. Das Standardmodell der Elementarteilchenphysik beschreibt als eine Theorie die starke, die schwache und die elektromagnetische Wechselwirkung, sowie alle bekannten Elementarteilchen.
Elementarteilchen sind der Grundbaustein von Materie und zeichnen sich dadurch aus, dass sie, nach aktuellem Wissensstand, nicht weiter teilbar sind.

Die Elementarteilchen teilen sich in zwei Gruppen auf, die Quarks und die Leptonen. Beide Gruppe beinhalten sechs Teilchen. Es gibt folgende sechs Quarks: \textit{up} ($u$), \textit{down} ($d$), \textit{strange} ($s$), \textit{charm} ($c$), \textit{bottom} ($b$) und \textit{top} ($t$) und folgende sechs Leptonen: Elektron ($e$), Elektron-Neutrino ($\nu_\text{e}$), Myon ($\mu$), Myon-Neutrino ($\nu_{\mu}$), Tau ($\tau$) und Tau-Neutrino ($\nu_{\tau}$). Beide Gruppen kann man weiter aufteilen in insgesamt drei Generationen. Die Generationen der Leptonen und Quarks h\"angen dabei nicht zusammen und die Teilchen sind mit der Masse aufsteigend den Generationen zugeordnet. In Tabelle \ref{tab:teilchen} sind die Elementarteilchen bez\"uglich ihrer Generation und Ladung sortiert aufgelistet. Zu den Elementarteilchen gibt es entsprechende Antiteilchen, mit umgekehrter Ladung, wie beispielsweise das Positron, das Antiteilchen des Elektrons mit einfach positiver elektrischer Ladung.
F\"ur diese Arbeit sind besonders das $u$ und $d$ Quark, sowie das Elektron und das Positron von Bedeutung.

Durch das tragen von Ladung sind Teilchen in der Lage an Wechselwirkung teilzunehmen. Dabei gilt es zwischen verschiedenen Ladungen zu unterscheiden. Die elektrische Ladung erm\"oglicht die Teilnahme an der elektrischen Wechselwirkung, die Farbladung hingegen die Teilnahme an der starken Wechselwirkung.

Die Elementarteilchen kann man nun entsprechend ihrer elektrischen Ladung und der Generation unterteilen. Ebenso lassen sich die Austauschteilchen ihrer entsprechenden Wechselwirkung zuordnen, wobei diese ebenfalls zu den Elementarteilchen z{\"a}hlen. Dies wurde in den beiden Tabellen \ref{tab:teilchen} und \ref{tab:Austeilchen} gemacht. Es gibt also sechs Quarks und sechs Leptonen, mit jeweils ihren entsprechend Antiteilchen.

\begin{table}[h] 
\centering
\begin{tabular}{|c||c|c|c||c|}
\hline
Generation & I                                                                                    & II                                                                              & III                                                                             & el. Ladung                                           \\ \hline \hline
Quarks    & \begin{tabular}[c]{@{}c@{}}up ($u$)\\ down ($d$)\end{tabular}                        & \begin{tabular}[c]{@{}c@{}}charm ($c$)\\ strange ($s$)\end{tabular}             & \begin{tabular}[c]{@{}c@{}}top ($t$)\\ bottom ($b$)\end{tabular}                & \begin{tabular}[c]{@{}c@{}}+2/3\\  -1/3\end{tabular} \\ \hline
Leptonen  & \begin{tabular}[c]{@{}c@{}}Elektron ($e$)\\ Elektron-Neutrino ($\nu_{e}$)\end{tabular} & \begin{tabular}[c]{@{}c@{}}Myon($\mu$)\\ Myon-Neutrino ($\nu_{\mu}$)\end{tabular} & \begin{tabular}[c]{@{}c@{}}Tau($\tau$)\\ Tau-Neutrino ($\nu_{\tau}$)\end{tabular} & \begin{tabular}[c]{@{}c@{}}-1\\  0\end{tabular}      \\ \hline
\end{tabular}
\caption{Elementarteilchen geordnet nach ihrer Generation und ihrer elektrische Ladung}
\label{tab:teilchen}
\end{table}


\begin{table}[h]
\centering
\begin{tabular}{|c||c|c|c|}
\hline
Wechselwirkung    & elektromagnetisch & stark       & schwach                      \\ \hline
Austauschteilchen & Photon ($\gamma$) & Gluon ($g$) & $W^{\pm}$, $Z^{0}$ - Bosonen \\ \hline
\end{tabular}
\caption{Tabelle der Austauschteilchen}
\label{tab:Austeilchen}
\end{table}

Zus\"atzlich zu den bisher genannten Gr\"o{\ss}en lassen sich den Elementarteilchen noch weitere Eigenschaften zuschreiben. So haben alle drei Austauschteilchen einen Spin von $1$ und geh\"ohren somit zu den Bosonen. Quarks und Leptonen tragen hingegen einen Spin von $1/2$ und geh\"ohren entsprechend zu den Fermionen.

Wie oben bereits erw\"ahnt koppeln Kr\"afte an die Ladung von Teilchen. Um die elektromagnetische Kraft zu Beschreiben gibt es die Quantenelektrodynamik (QED). F\"ur die starke Wechselwirkung gibt es entsprechend die Quantenchromodynamik (QCD). Der Name kommt daher, dass die Ladung, die zur starken Wechselwirkung geh\"ohrt, die Farbladung ist (chroma ist griechisch für Farbe). Die Farbladung hat dabei nichts mit der \"au{\ss}eren Erscheinung der Quarks zu tun.