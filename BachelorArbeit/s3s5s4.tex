\begin{figure}[t!]
\centering
\includegraphics[width=.65\linewidth]{ParamResult_Bin10_Data_2016.pdf}
\caption{Signal mit unkorreliertem Untergrund zusammen mit Parametrisierung der Templates des korrelierten Untergrund und des Signals.
}
\label{fig:ParamResult}
\end{figure}
Das Ergebnis der Parametrisierung der Templates für das $p_\text{T}$-Intervall $(3\,2-3\,4)(\text{GeV/}c)$ wird in Abbildung \ref{fig:ParamResult} dargestellt.
Die Parametrisierung der beiden Templates stimmt innerhalb der Unsicherheiten gut mit den Daten überein, wie nach Abbildung \ref{fig:Chi2pT} zu erwarten war.
\newline
Um die Anzahl produzierter $\pi^{0}$ nun zu bestimmen, wird das skalierte Template des korrelierten Untergrunds von dem Signal ohne kombinatorischen Untergrund abgezogen.
Anschließend wird in einem bestimmten Zählbereich über die Werte des Signals integriert.
Dies wird für jedes $p_\text{T}$-Intervall durchgeführt.
\newline
Der Zählbereich liegt fest bei $0\,06 \leq m_\text{inv}/\text{ (GeV}/c^{2}) < 0\,25$.
In Abbildung \ref{fig:ParamResult} wird er Zählbereich durch eine blaue Linie markiert.
\begin{figure}[t!]
\centering
\includegraphics[width=.65\linewidth]{UncorrYields_Data_2016.pdf}
\caption{Anzahl rekonstruierter $\pi^{0}$ in Abhängigkeit von $p_\text{T}$.
}
\label{fig:RawYield}
\end{figure}
\newline
Das so erhaltene rohe Spektrum wird zusätzlich noch auf die Anzahl an \textit{Events} $N_\text{evt}$, den Pseudorapiditätsbereich $\eta$, den Transversalimpuls $p_\text{T}$, die Wahrscheinlichkeit, dass ein $\pi^{0}$ in zwei Photonen zerfällt und $2\pi$ normiert.
Letzteres ist reine Konvention, während die anderen Normierungen für einen Vergleich zwischen unterschiedlichen Analysen benötigt werden.
Abbildung \ref{fig:RawYield} zeigt das normierte rohe Spektrum.
Das Spektrum steigt zunächst leicht an, bis es bei $1\,8 \leq p_{\text{T}}/(\text{ GeV}/c) < 2\,0$ sein Maximum erreicht.
Danach sinkt das Spektrum kontinuierlich.
\newline
Für eine Aussage über die Produktionsrate von $\pi^{0}$ sowie einen allgemeinen Vergleich wird allerdings noch die Korrektur auf die Detektorakzeptanz sowie die Rekonstruktionseffizienz benötigt.