Nach heutigem Wissensstand begann das Universum mit dem Urknall.
Kurz nach dem Urknall lag die Materie des Universums in einem heißen und dichten Zustand vor, der als Quark-Gluonen-Plasma (QGP) bezeichnet wird.
Quarks und Gluonen sind elementare Bausteine, die den Grundstein für jegliche uns heute umgebende Materie bilden.
Heute liegen die Quarks und Gluonen in der Natur in Form von gebundenen Zuständen vor, wie etwa dem Proton und dem Neutron.
\newline
Für die Bindung der Quarks und Gluonen ist die starke Wechselwirkung, eine der vier fundamentalen Kräfte der Physik, verantwortlich.
Um die starke Wechselwirkung untersuchen zu können werden quasi freie Quarks und Gluonen benötigt.
Innerhalb des QGP, das bei hohen Energiedichten auftritt, können Quarks und Gluonen sich quasi frei bewegen.
\newline
Um ein Quark-Gluonen-Plasma zu erzeugen, werden Blei-Ionen auf fast Lichtgeschwindigkeit beschleunigt und anschließend zum Kollidieren gebracht.
Solche Kollisionen geschehen in Beschleunigeranalagen wie dem \textit{Large Hadron Collider} (LHC) am Kernforschungsinstitut CERN.
Eines der vier großen Experimente am LHC ist das ALICE Experiment, das speziell zur Untersuchung des QGP gebaut wurde.
\newline
Neben Blei-Ionen werden auch Protonen dort zur Kollision gebracht, bei denen davon ausgegangen wird, dass kein QGP entsteht.
In beiden Kollisionen entstehen viele verschiedene Teilchen, wie das $\pi^{0}$.
Die Produktionsrate eines Teilchen aus Proton-Proton-Kollisionen kann dann mit der Produktionsrate des selben Teilchens in Blei-Blei-Kollisionen verglichen werden, um so Eigenschaften des QGP bestimmen zu können.
\newline
Entstandene Teilchen am ALICE Experiment werden mit verschiedenen Detektoren wie dem elektromagnetischen Kalorimeter (EMCal) detektiert.
Das $\pi^{0}$ kann dabei jedoch nicht selbst direkt gemessen werden, da es sehr schnell zerfällt.
Mit einer Wahrscheinlichkeit von circa $99\%$ zerfällt das $\pi^{0}$ in zwei Photonen, deren Energie und Position mit dem EMCal gemessen werden können.
Um die Produktionsrate von $\pi^{0}$ in einem Kollisionsexperiment bestimmen zu können, muss das Signal aus Photonenpaaren von $\pi^{0}$-Zerfällen extrahiert werden.
\newline
In dieser Arbeit wird eine Analysemethode vorgestellt in der die Anzahl produzierter $\pi^{0}$ in Proton-Proton-Kollisionen mit Hilfe von Templates aus Monte Carlo Simulationen bestimmt wird.
Die dafür verwendeten Daten stammen aus Proton-Proton-Kollisionen, die im Jahr 2016 bei einer Schwer\-punkts\-en\-er\-gie von $\sqrt{s}=13$ TeV stattfanden.
\newline
Die Arbeit lässt sich dabei in vier Abschnitte einteilen.
Im ersten Abschnitt werden die physikalischen Grundlagen näher dargelegt, die für diese Arbeit notwendig sind.
Der Aufbau des ALICE Experiments sowie eine kurze Beschreibung der für diese Analyse verwendeten Detektoren mit Fokus auf dem EMCal werden in Abschnitt zwei erläutert.
Die Analyse mit Hilfe von Monte Carlo Templates wird im anschließenden ausführlich vorgestellt und im  abschließenden mit der bisher gängigsten Analysemethode verglichen.