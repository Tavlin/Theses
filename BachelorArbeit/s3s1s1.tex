In dieser Arbeit werden Daten von pp-Kollisionen bei $\sqrt{s}=13\text{ TeV}$ verwendet, die mit dem ALICE Experiment gemessen wurden.
Die Datensätze werden unterteilt in Perioden, die ungefähr einem Monat Aufnahmezeit entsprechen.
Dabei existiert eine Namenskonvention für die Perioden.
Sie beginnen immer mit LHC, gefolgt von den zwei hinteren Jahreszahlen in der die Periode gemessen wurde.
Zuletzt noch ein Kleinbuchstabe der die Perioden sortiert, beginnend bei a.
Die Perioden die in dieser Analyse verwendet werden sind $LHC16h\,i\,j\,k\,l$.
Diese umfassen circa 250 Millionen \textit{minimum-bias events}.
\newline
Für die Monte Carlo Simulation wurde der Ereignisgenerator PYTHIA 8 mit dem \textit{tune} Monash 2013 benutzt.
Dabei sei erneut auf \cite{thesis:Krissy} verwiesen.
Außerdem wurde GEANT3 verwendet um die möglichen Interaktionen mit dem ALICE Experiment zu simulieren \cite{Brun:118715}.
Die Monte Carlo Simulation wurde dabei an die  Perioden $LHC16h\,i\,j\,k\,l$ angepasst.
Insgesamt umfasst die Monte Carlo Simulation etwa 280 Millionen \textit{minimum-bias events}.