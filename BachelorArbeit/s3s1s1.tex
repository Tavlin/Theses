In der in dieser Arbeit vorgestellten Analyse werden Daten von pp-Kollisionen bei $\sqrt{s}=13\text{ TeV}$ verwendet, die mit dem ALICE Experiment gemessen wurden.
Datensätze des ALICE Experiments werden in Perioden unterteilt, die ungefähr einem Monat Aufnahmezeit entsprechen.
Für die Perioden gibt es eine Namenskonvention: LHC[Jahr][Perioden-Index].
Das Jahr wird dabei nicht vollständig angegeben, sondern nur die letzten beiden Ziffern der Jahreszahl.
Bei dem Perioden-Index handelt es sich um einen Kleinbuchstaben.
Er sortiert die Perioden aufsteigend, beginnend bei a.
Die Perioden die in dieser Analyse verwendet werden sind LHC16h,i,j,k,l.
Diese umfassen zusammen circa 250 Millionen \textit{minimum-bias Events}.
\newline
Für die Monte Carlo Simulation wurde der Ereignisgenerator PYTHIA 8 mit dem \textit{Tune} Monash 2013 benutzt \cite{thesis:Krissy}.
Außerdem wurde GEANT3 verwendet um die möglichen Interaktionen der in der Kollision entstandenen Teilchen mit dem ALICE Experiment zu simulieren \cite{Brun:118715}.
Die Monte Carlo Simulation wurde dabei an die  Perioden LHC16h,i,j,k,l angepasst.
Insgesamt umfasst die Monte Carlo Simulation etwa 280 Millionen \textit{minimum-bias Events}.