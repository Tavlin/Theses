Die Korrekturen die auf das rohe Spektrum angewandt werden basieren auf der Monte Carlo Simulationen, aus der auch die Templates für diese Analyse stammen.
\newline
Die Detektorakzeptanz spiegelt dabei die Abdeckung des EMCals wider.
Sie berechnet sich aus dem Verhältnis der Photonenkandidatenpaare, die auf das EMCal treffen, zu den produzierten Photonenkandidatenpaare.
\newline
Die Effizienz berechnet sich aus der Division der Photonenkandidatenpaare aus dem Template des Signals geteilt durch die Anzahl der akzeptierten Photonenkandidatenpaare.
Für die Effizienz wird der $m_\text{inv}$ Bereich zum Zählen benutzt, der auch für die Bestimmung des rohen Spektrums benutzt wurde.
\newline
Durch das Korrigieren des rohen Spektrums mit der Detektorakzeptanz und der Effizienz, wird die Anzahl der detektierten und extrahierten $\pi^{0}$ auf die Anzahl der produzierten $\pi^{0}$ korrigiert.
Abbildung \ref{fig:Korrekturen} zeigt die beiden Korrekturgrößen Detektorakzeptanz und Effizienz.
%%%text was was ist und wie es verläuft und warum
Abbildung \ref{fig:CorrYield} zeigt den korrigierten Yield.
%%%text zum verlauf und aussehen?