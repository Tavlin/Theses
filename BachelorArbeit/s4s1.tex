Die Korrekturen, die auf das rohe Spektrum angewandt werden, basieren auf der Monte Carlo Simulation aus der auch die Templates für diese Analyse stammen.
\newline
Die Detektorakzeptanz spiegelt dabei die räumliche Abdeckung des EMCals wider.
Sie berechnet sich aus dem Verhältnis der \textit{Clusterpaare}, die auf das EMCal treffen, zu den produzierten \textit{Clusterpaaren}.
\newline
Die Rekonstruktionseffizienz berechnet sich aus der Division der \textit{Clusterpaare} aus dem Template des Signals durch die Anzahl der akzeptierten \textit{Clusterpaare}.
Für die Rekonstruktionseffizienz wird der $m_\text{inv}$-Bereich zum Zählen benutzt, der auch für die Bestimmung des rohen Spektrums benutzt wurde.
\begin{figure}[t] %kein t!
\centering
\includegraphics[width=.65\linewidth]{Korrekturfaktoren_Data_2016.pdf}
\caption{Detektorakzeptanz und Rekonstruktionseffizienz in Abhängigkeit von $p_\text{T}$.
}
\label{fig:Korrekturen}
\end{figure}
\newline
Durch das Korrigieren des rohen Spektrums mit der Detektorakzeptanz und der Rekonstruktionseffizienz, wird die Anzahl der rekonstruierten $\pi^{0}$ auf die Anzahl der produzierten $\pi^{0}$ korrigiert.
Abbildung \ref{fig:Korrekturen} zeigt die beiden Korrekturgrößen Detektorakzeptanz und Rekonstruktionseffizienz.
\newline
Die Detektorakzeptanz wird mit steigendem $p_\text{T}$ größer.
Zerfällt ein $\pi^{0}$ in zwei Photonen, so fliegen die Photonen im System des $\pi^{0}$ entgegengesetzt zu einander weg, der Öffnungswinkel zwischen den beiden Photonen $\theta_{\gamma\gamma}$ beträgt in diesem System $180^{\circ}$.
Abhängig vom $p_\text{T}$ den ein $\pi^{0}$ im Laborsystem hat verringert sich $\theta_{\gamma\gamma}$.
Je größer $p_\text{T}$ umso kleiner $\theta_{\gamma\gamma}$.
Deshalb treffen beide Photonen aus einem $\pi^{0}$-Zerfall bei niedrigem $p_\text{T}$ seltener auf das EMCal.
Aus diesem Grund steigt die Detektorakzeptanz mit wachsendem $p_\text{T}$ leicht an.
\newpage
\noindent
Die Rekonstruktionseffizienz steigt bis $p_\text{T} = 8 \text{ GeV}/c$ an und saturiert dann.
Durch die Anforderungen an die \textit{Cluster} werden unter anderem auch \textit{Cluster} ausgeschlossen, deren zugrunde liegende Teilchen aus dem Zerfall eines $\pi^{0}$ stammen, aber beispielsweise zu wenig Energie besitzen.
Gerade asymmetrische Zerfälle, bei denen der Anteil der zur Verfügung stehenden Energie ungleichmäßig auf die Zerfallsprodukte aufgeteilt wird, werden durch die Energieanforderung an die \textit{Cluster} ausgeschlossen.
Im Saturationsbereich werden immer mehr \textit{Cluster} zusammengefasst, da sie zu nah aneinander liegen.
Dadurch alleine würde die Rekonstruktionseffizienz wieder sinken, jedoch liegt der vorher genannte Effekt in diesem Bereich auch vor, wodurch die Saturation entsteht.
\begin{figure}[t!]
\centering
\includegraphics[width=.65\linewidth]{KorrigierterYieldNurStat_Data_2016.pdf}
\caption{Korrigiertes $p_\text{T}$-Spektrum mit statistischer Unsicherheit.
}
\label{fig:YieldStatUncer}
\end{figure}
\newline
Die Detektorakzeptanz und die Rekosntruktionseffizienz werden auf das $p_\text{T}$-Spektrum der rekonstruierten $\pi^{0}$ angewandt.
Abbildung \ref{fig:YieldStatUncer} zeigt das korrigierte $p_\text{T}$-Spektrum, das nun die Anzahl der produzierten $\pi^{0}$ darstellt.
Ausgehend von dem korrigierten $p_\text{T}$-Spektrum wird im folgenden Abschnitt die systematische Unsicherheit für dieses bestimmt.