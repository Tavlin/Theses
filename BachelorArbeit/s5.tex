In der vorliegenden Arbeit werden Daten von pp-Kollisionen bei $\sqrt{s}=13\text{ TeV}$ analysiert.
Die Daten sind im Jahr 2016 mit dem ALICE-Experiment aufgezeichnet worden.
Bei der Analyse handelt es sich um eine Analyse neutraler $\pi^{0}$ aus Messungen des EMCals.
\newline
Zunächst werden die Zellen des EMCals, die Energie gespeichert haben, zu \textit{Clustern} zusammengefasst.
Um \textit{Cluster}, denen Photonen zugrunde liegen, von anderen \textit{Clustern} zu unterscheiden, werden verschiedene Anforderungen an die \textit{Cluster} gestellt.
Die so selektierten \textit{Clustern} werden anschließend in der \textit{same Event} Methode paarweise miteinander kombiniert und der Transveralsimpuls sowie die invariante Masse der \textit{Clusterpaare} berechnet.
\newline
Die daraus entstandene Verteilung der invarianten Masse und des Transversalimpulses besteht aus rekonstruierten $\pi^{0}$ und Untergrund.
Der Untergrund teilt sich in zwei Komponenten auf, den korrelierten und den unkorrelierten Untergrund.
Um den unkorrelierten Untergrund abzuschätzen wurden \textit{Cluster} aus unterschiedlichen Events in der \textit{mixed Event} Methode miteinander kombiniert und skaliert.
\newline
Im Rahmen dieser Analyse werden die Verteilungen in 35 $p_\text{T}$-Intervalle aufgeteilt und einzeln betrachtet.
Die skalierte Verteilung der invarianten Masse aus der \textit{mixed Event} Methode wird von der Verteilung der invarianten Masse aus der \textit{mixed Event} abgezogen.
\newline
Die verbleibende Verteilung wird in dieser Arbeit durch Templates beschrieben.
Die Templates werden mit Hilfe einer Monte Carlo Simulation erzeugt.
Für jedes $p_\text{T}$-Intervall wird dabei ein eigenes Template des Signals verwendet, während es ein global benutztes Template des korrelierten Untergrunds gibt.
Das Template des korrelierten Untergrunds wird mit Hilfe einer speziell für diesen Zweck angepassten Monte Carlo Simulation an die einzelnen $p_\text{T}$-Intervalle angepasst.
Diese Anpassung wird benötigt, da die Anforderung an den Öffnungswinkel zwischen zwei Photonen in einer $p_\text{T}$ abhängige untere Grenze für $m_\text{inv}$ resultiert.
\newline
Um die Templates an die Daten anzupassen werden diese mittels $\chi^{2}$-Minimierung parametrisiert.
Das skalierte Template des korrelierten Untergrunds, das aus der Parametrisierung kommt, wird von der Verteilung der invarianten Masse der Daten ohne unkorrelierten Untergrund subtrahiert.
Die übrige Verteilung der Daten, das Signal wird anschließend über einen festen $m_\text{inv}$-Bereich integriert und damit die Anzahl rekonstruierter $\pi^{0}$ in Abhängigkeit von $p_\text{T}$ bestimmt.
\newline
Das so extrahierte $p_\text{T}$-Spektrum wird anschließend auf die Detektorakzeptanz und die Rekonstruktionseffizienz korrigiert.
Durch Variationen in der Signalextraktion wird anschließend die systematische Unsicherheit im korrigierten $p_\text{T}$-Spektrum bestimmt.
\newline
Abschließend wird das korrigierte $p_\text{T}$-Spektrum aus der Analyse mit Hilfe von Monte Carlo Templates verglichen mit dem korrigierten $p_\text{T}$-Spektrum der Analyse mit Hilfe von Funktionsparametrisierung.
Die Abweichung der beiden Spektren liegt im $5\%$ Bereich und das Verhältnis der beiden Spektren ist innerhalb der statistischen Unsicherheit einigermaßen mit $1$ vereinbar.
Zuletzt wird die statistische Unsicherheit der beiden Analysemethoden verglichen.
Dabei zeigt sich, dass die Analyse mit Hilfe von Monte Carlo Templates fast in allen $p_\text{T}$-Intervallen geringere statistische Unsicherheiten hat, als die andere Analysemethode.
\newline
Für einen vollständigen Vergleich der beiden Methoden wird noch die systematische Unsicherheit der Analyse mit Funktionsparametrisierung benötigt.
Außerdem kann die Analyse mit Tempaltes noch erweitert werden, indem der unkorrelierte Untergrund ebenfalls durch die Verwendung eines Templates zusammen mit dem Template des Signals und dem Templates des korrelierten Untergrunds parametrisiert wird.